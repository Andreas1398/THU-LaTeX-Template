\chapter{Das dritte Kapitel}

\section{Einstellungen}
In der Datei /content/settings.tex müssen dokumentspezifische Werte wie beispielsweise der Name des Autors und der Titel der Arbeit bearbeitet werden. Die Parameter sind in der Datei ebenfalls erklärt. Zu weiteren Beschreibungen kann in der README.md nachgeschaut werden.

Es wird empfohlen, die Output Parameter erst am Ende zu definieren, um so nicht benutzte Seiten herauszunehmen. Damit wird verhindert, dass Content nicht richtig generiert werden kann, wenn beispielsweise die Definition von Abkürzungen fehlt.

\subsubsection{Hinweis:}
Das Glossar wird, wenn es leer ist, lediglich als leere Seite ohne entsprechende Überschrift angezeigt.


\section{Vorwort und Abstract}
Das Vorwort kann in der Datei /content/specialPages/Foreword.tex geschrieben werden. Der Befehl \texttt{\textbackslash{}signature} fügt dabei den Block für die Unterschrift(en) ein.

Das Abstract wird in den deutschen Einstellungen sowohl auf Deutsch als auch auf Englisch ausgegeben. Aus diesem Grund muss man hierfür sowohl die Datei Abstract\_en.tex als auch die Datei Abstract\_de.tex im Ordner /content/specialPages/ bearbeiten. Dazu gehört die Definition der Keywords, welche nicht weggelassen werden kann.

In den englischen Einstellungen muss lediglich die Datei Abstract\_en.tex bearbeitet werden.