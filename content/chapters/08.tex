\chapter{The fourth chapter}

\section{Compiling the LaTeX template}
This template can be edited and used with different programs. Local programs such as TeXstudio or Visual Studio Code with the right plugins can be used to compile the template.

However, it is recommended to use the free online editor Overleaf \href{https://www.overleaf.com/}{https://www.overleaf.com/}. This is easier to set up, enables sharing and collaborative editing of the document, saves the history and much more. However, it is essential to clarify the confidentiality of the work with examiners and supervisors, as this would speak \textbf{against using Overleaf}. Overleaf stores all uploaded content on their servers, which could mean a breach of confidentiality.

No matter which tool is used: In general, the file \textbf{main.tex} must be compiled. The compiler \textbf{pdfLaTeX} is recommended. It may also work with other compilers, but there is no guarantee that everything will work correctly!


\section{Spelling and grammar checker}
To use a spell and grammar checker when writing, it is recommended to use a tool such as LanguageTool (\href{https://languagetool.org/}{https://languagetool.org/}). This can be set up as an add-on in the browser for use in Overleaf, but can also be installed locally for local use with Visual Studio Code or TeXstudio, for example. A list of possible applications can be found here: \href{https://dev.languagetool.org/software-that-supports-languagetool-as-a-plug-in-or-add-on.html}{https://dev.languagetool.org/software-that-supports-languagetool-as-a-plug-in-or-add-on.html}.

\subsubsection{IMPORTANT:}
However, it is essential to clarify the confidentiality of the work with examiners and supervisors before using it, as this would speak \textbf{against the use of LanguageTool}. LanguageTool uploads all text to their servers to check it there. This may constitute a breach of confidentiality.

(Exception: When used locally, for example as a plug-in in Visual Studio Code, offline use is also possible.)