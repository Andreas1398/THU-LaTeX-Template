\chapter{Das vierte Kapitel}

\section{Kompilieren der LaTeX-Vorlage}
Diese Vorlage kann mit unterschiedlichen Programmen bearbeitet und genutzt werden. Hier können lokale Programme wie TeXstudio oder Visual Studio Code mit den richtigen Plugins genutzt werden, um die Vorlage zu kompilieren.

Es wird jedoch empfohlen, den kostenlosen Online-Editor Overleaf \href{https://www.overleaf.com/}{https://www.overleaf.com/} zu nutzen. Dieser ist leichter einzurichten, ermöglicht das Teilen und das gemeinsame Bearbeiten des Dokuments, speichert die Historie und vieles mehr. Es ist jedoch unabdinglich, die Vertraulichkeit der Arbeit mit Prüfern und Betreuern abzuklären, da dies \textbf{gegen eine Nutzung von Overleaf} sprechen würde. Overleaf speichert sämtliche hochgeladenen Inhalte auf deren Servern, was gegebenenfalls eine Verletzung der Vertraulichkeit bedeuten könnte.

Egal, welches Tool benutzt wird: Generell muss die Datei \textbf{main.tex} kompiliert werden. Empfohlen wird der Compiler \textbf{pdfLaTeX}. Mit anderen Compilern kann es ebenfalls funktionieren, aber keine Garantie, dass auch alles richtig funktioniert!

\section{Rechtschreib- und Grammatikprüfung}
Um beim Schreiben eine Rechtschreib- und Grammatikprüfung zu nutzen, empfiehlt es sich, ein Tool wie beispielsweise LanguageTool (\href{https://languagetool.org/}{https://languagetool.org/}) zu benutzen. Dieses lässt sich für die Nutzung in Overleaf als Add-on im Browser einrichten, für die lokale Nutzung mit beispielsweise Visual Studio Code oder TeXstudio jedoch auch lokal installieren. Eine Liste der Anwendungsmöglichkeiten findet sich hier: \href{https://dev.languagetool.org/software-that-supports-languagetool-as-a-plug-in-or-add-on.html}{https://dev.languagetool.org/software-that-supports-languagetool-as-a-plug-in-or-add-on.html}.

\subsubsection{WICHTIG:}
Für dessen Nutzung ist es jedoch auch hier unabdinglich, die Vertraulichkeit der Arbeit mit Prüfern und Betreuern abzuklären, da dies \textbf{gegen eine Nutzung von LanguageTool} sprechen würde. LanguageTool lädt sämtlichen Text auf deren Server, um ihn dort zu überprüfen. Dies kann gegebenenfalls eine Verletzung der Vertraulichkeit bedeuten.

(Ausnahme: Bei der lokalen Nutzung, beispielsweise als Plugin in Visual Studio Code, ist auch eine offline Benutzung möglich.)